\documentclass{article}
\usepackage{amssymb,amsmath,indentfirst}
\usepackage{todonotes}
\usepackage{graphicx}
\newcommand{\entities}{{\cal E}}
\newcommand{\ent}{e}
\newcommand{\roles}{{\cal R}} 
\newcommand{\role}{r}
\newcommand{\Tv}{T}
\newcommand{\tv}{t}
\newcommand{\sent}{s}
\newcommand{\weight}{\mathbf{w}}
\newcommand{\rend}{\rho}
\newcommand{\Rend}{{\cal P}}
\newcommand{\trans}{q}
\newcommand{\obj}{f}
\newcommand{\commentout}[1]{}
\newcommand{\Enum}[1]{\{1 \ldots #1\}}
\newcommand{\EnumS}[2]{\{#1 \ldots #2\}}
\newcommand{\Set}[1]{\big\{ #1 \big\}}
\newcommand{\Bin}[1]{\{0,1\}^{#1}}
\DeclareMathOperator*{\argmax}{arg\,max}
\DeclareMathOperator*{\argmin}{arg\,min}
\newcommand{\rot}[1]{\rotatebox{90}{#1}} 
\newcommand{\fmap}{\boldsymbol{\phi}}
\newcommand{\edgeTot}{\boldsymbol{\psi}}

\begin{document}
\section{Introduction}
Automatic coherence assessment has typically examined the effect of local discourse features on coherence. These features are typically sentence-to-sentence transitions of entities, discourse relations, or other sentential information. 
Additionally, this task is often framed as selecting the most coherent ordering of a set of sentences in a document, where the original ordering as written by a human is assumed to be the most coherent ordering.\\\\


\todo[inline]{related work}

\section{Entity Grid Model}

The {\em entity grid} model is a popular local coherence model where a document rendering is represented by a matrix or grid of discourse entities and their syntactic role, such as the \textit{subject}, \textit{object}, etc. 
Each entity in the document has a column and each sentence has a row in the grid. 
A role transition is simply a column subsequence of any length. For example, in Table ~\ref{tab:egrid} the entity \textit{winds} has a length 2 transition \{-,o\} in sentences 2--3. 


\begin{table}[h]
\begin{center}
    \begin{tabular}{ l | l  l  l  l  l  l  l  l  l  l  l  }
    & \rot{Karen}  & \rot{winds}  & \rot{Friday} & \rot{Wash., D.C.} & \rot{Monday} & \rot{Gulf of Mexico} & \rot{coast} & \rot{weekend} & \rot{authorities} & \rot{orders} & \rot{New Orleans} \\ \hline
    $\mathbf{1}$ & s & - & - & - & - & x & x & x & - & - & - \\ 
    $\mathbf{2}$ & s & - & x & - & - & - & x & - & s & o & x \\ 
    $\mathbf{3}$ & s & o & x & - & - & - & - & - & - & - & - \\ 
    $\mathbf{4}$ & s & - & - & x & x & - & - & - & - & - & - \\ 
    \end{tabular}
\end{center}
\caption[Entity grid representation]{An example entity grid representation.}
\label{tab:egrid}
\end{table}

Let a document $d$ be an undordered set of $n$ sentences. 
We define the set $\entities = \{e_1,e_2,\ldots,e_{|\entities|}\}$ to be the set of $|\entities|$ entities that occur in $d$ and the set $\roles$ to be a finite set of syntactic roles. 
We also define the set of document renderings $\Rend = \{\rend_1,\rend_2,\ldots, \rend_{n!} \}$ to be the set of ordered permutations of these sentences. 
A document rendering $\rend_i = \{\sent_{i,j} : 1 \le j \le n \}$ is an ordered sequence of sentences where a sentence is a vector $\sent_{i,j} = \{\role_{i,j,\ent_k} : 1\le k \le |\entities| \}$ of syntactic roles.


We formally define the entity grid model of a document rendering $\rend_i$ to be the sets $\entities$, $\roles$, and the set of transition vectors $\tv_{i,j}$  for all $j$, $2 \le j \le n$. 
A transition vector $\tv_{i,j} \in \{\roles\times\roles\}^{|\entities|}$ is a vector of entity role transitions from $\sent_{i,j-1}$ to $\sent_{i,j}$ where each element corresponds to a transition  $\trans \in \{\roles\times\roles\}$.
Define the function $\fmap: \Rend \rightarrow \mathbb{R}^{| \roles \times \roles | }$ with $\fmap(\rend_i) = \{ \phi_\trans(\rend_i) : \forall \trans \in {\roles \times \roles}  \}$ and

\begin{equation}
\phi_\trans(\rend_i) = \frac{ \displaystyle \sum_{j = 2}^n \sum_{k=1}^{|\entities|} I(\tv_{i,j,e_k}= \trans)  }{|\entities|(n-1)}. 
\end{equation}


Given a vector of weights $\mathbf{w} \in \mathbb{R}^{|\roles\times\roles|}$, we define our objective function $\obj : \mathbb{R}^{|\roles\times\roles|}  \rightarrow \mathbb{R}$ where $\obj(\fmap(\rend_i);\weight) = \weight\cdot\fmap(\rend_i)$. 
The sentence ordering task can further be defined as finding the rendering that maximizes this function, or

\begin{equation}
    \argmax_{i \in \EnumS{1}{n!}} \obj(\boldsymbol{\phi}(\rend_i) ; \weight ).
\end{equation}


Solving this problem for larger documents is intractable as the time complexity of computing the $\argmax$ is $ O(|\Rend||\roles|^2) = O(n!|\roles|^2)$; to compute each ranking, we must perform $|\roles|^2$ multiplications, and to find the maximum we must check $n!$ possible renderings.

\section{TSP Formulation}
We formulate this problem as a Traveling Salesperson Problem where each sentence $\sent$ is a vertex in a fully connected graph. 
Adding an edge $(\sent_i,\sent_j)$ to the solution path represents selecting $\sent_j$ as the next sentence in the document rendering.
In order to calculate edge weights, we define a function $\edgeTot : E \rightarrow \mathbb{Z}^{|\roles \times \roles|}  $ where $\edgeTot(s_i,s_j) = \{ \psi_\trans(s_i,s_j) : \trans \in {\roles \times \roles } \}$ and 
\begin{equation}
    \psi_\trans\left( s_i,s_j \right) = \displaystyle \sum_{k=1}^{|\entities|} I\left( \langle s_{i,\ent_k},s_{j,\ent_k} \rangle = \trans\right) . 
\end{equation}

Edge weights are determined by the function

\begin{equation}
    g(\edgeTot(\sent_i,\sent_j );\weight) = \weight\cdot \edgeTot(\sent_i, \sent_j) = \sum^{|\roles|^2}_{q=1} w_\trans \left( \sum_{k = 1}^{|\entities|} I(\langle \sent_{i,\ent_k},\sent_{j,\ent_k}\rangle = \trans) \right).
\end{equation}

where $\langle \sent_{i,\ent_k},\sent_{j,\ent_k} \rangle$ is a transition $q \in \{\roles \times \roles\}$ for entity $k$  obtained from ordered sequence of $\sent_i,\sent_j$.


Under this formulation, the optimal path is equivalent to the solution to the optimal rendering in the entity grid formulation. 
The total edge weights for the optimal path are the sum of $g(\edgeTot(\sent_i,\sent_j);\weight)$ for all $(\sent_i,\sent_j)$ in the path.
Without loss of generality, let the optimal path $\rend = \{ \sent_1,\sent_2,\ldots,\sent_n \}$.

\begin{equation*}
\begin{split}
\textrm{Optimal Path Edge Weight} &= \sum^{n}_{j=2} g(\edgeTot(\sent_{j-1},\sent_j );\weight) \\
& = \sum^n_{j=2} \sum^{|\roles|^2}_{\trans=1} w_\trans \left( \sum_{k = 1}^{|\entities|} I(\langle \sent_{j-1,\ent_k},\sent_{j,\ent_k}\rangle = \trans)  \right)\\
& = \sum^{|\roles|^2}_{\trans=1} w_\trans \left( \sum^n_{j=2}  \sum_{k = 1}^{|\entities|} I(\langle \sent_{j-1,\ent_k},\sent_{j,\ent_k}\rangle = \trans)  \right)\\
%& = \sum^\rend_{t} \sum^{|\roles|^2}_{i=1} w_i \Big( \sum_{j\in1,\ldots,|\entities|} I(\tv_{j}= r_i) \Big)\\
%& = \sum^{|\roles|^2}_{i=1} w_i \Big( \sum^\rend_{t} \sum_{j\in1,\ldots,|\entities|} I(\tv_{j}= r_i) \Big)\\
\end{split}
\end{equation*}

Equivalently the $\argmax$ of the Entity Model objective function is


\begin{equation*}
\begin{split}
    \argmax_{i \in \EnumS{1}{n!}} \obj(\fmap(\rend_i) ; \weight ) &= \argmax_{i \in \EnumS{1}{n!}} \sum^{|\roles|^2}_{\trans=1} w_\trans \left(  \frac{ \displaystyle \sum_{j = 2}^n \sum_{k=1}^{|\entities|} I(\tv_{i,j,e_k}= \trans)  }{|\entities|(n-1)} \right)\\    
    &= \argmax_{i \in \EnumS{1}{n!}} \sum^{|\roles|^2}_{\trans=1} w_\trans \left( \displaystyle \sum_{j = 2}^n \sum_{k=1}^{|\entities|} I(\tv_{i,j,e_k}= \trans)\right)\\    
    &= \argmax_{i \in \EnumS{1}{n!}} \sum^{|\roles|^2}_{\trans=1} w_\trans \left( \displaystyle \sum_{j = 2}^n \sum_{k=1}^{|\entities|}I(\langle\sent_{i,j-1,\ent_k},\sent_{i,j,\ent_k} \rangle   = \trans)\right)\\    
%\argmax_{\rend \in \Rend} \obj(\boldsymbol{\phi}_\rend ; \weight ) & =\operatorname{arg} \max_{\rend \in \Rend}\sum^{|\roles|^2}_{i=1} w_i \Big( \frac{\sum_{\tv}^\rend\sum_{j\in1,\ldots,|\entities|} I(\tv_{j}= r)  }{|\entities|(n-1)}   \Big)\\
%&  =\argmax_{\rend \in \Rend}\sum^{|\roles|^2}_{i=1} w_i \Big( \sum_{\tv}^\rend\sum_{j\in1,\ldots,|\entities|} I(\tv_{j}= r)     \Big)\\
\end{split}
\end{equation*} 

\todo{give an example. NetworkX}


\commentout{
of probabilities for each transition $\trans$ in the document where each element $\phi(r)$ is defined as  

The sentence $\sent_{i,j} = \{ \role_{i,j,\ent_1}, \role_{i,j,\ent_2},\ldots,\role_{i,j,\ent_{|\entities|}} \}$. 
The syntanctic role $r_{i,j,e_k}$ is the syntactic role of the $k^{\textrm{th}}$ entity in $\sent_{i,j}$.


The sentence $s_{i,j}$ is the $j^{\textrm{th}}$ sentence of document rendering $\rend_i$ for all $j$, $1 \le j \le n$.


A document rendering $\rend$ is an ordered sequence of sentences $\sent_1, \ldots, \sent_n$.
The possible set of renderings $\Rend$ consists of all permutations of these sentences and has $n!$ elements.

A sentence $\sent$ is a vector of length $|\entities|$ with elements $\role \in \roles$.
The $j^{\textrm{th}}$ element of $\sent_i$ corresponds to the sytnactic role $\role_{i,j}$ of entity $e_j$ in sentence $i$.
Ignore this junk\\
Let $\entities$ be the set of entities in a document.
Let $\roles$ be the set of possible roles an entity can have.\\
Let $\roles^2$ be the set of transitions.\\
$\trans_i \in \{1,\ldots, |\roles|^2\}^{|\entities|}$\\
$\sent_i \in \roles^{|\entities|}\;\forall i \in \{1,\ldots, n\}$ where $n$ is the length of the document.\\
$\sent_{i-1} = [-,-,\times,-]$\\ 
$\sent_{i} = [s,-,o,\times]$\\
$\trans_i = [(-,\times),(-,-),(\times,\times),(-,\times)]$\\ 
$\forall r \in |\roles|^2$, 
$\weight \in \mathbb{R}^{|\roles|^2}$
$f(\trans;\weight) = \weight \cdot \phi(\trans)$\\
$\max_{\trans \in \Tv} f(\trans ; \weight )$\\Y
Solving this is $O(M\roles^2) \Rightarrow O(n!\roles^2)$
}

\end{document}

%%% Local Variables: 
%%% mode: latex
%%% TeX-master: t
%%% End: 
