\documentclass[11pt]{article}
\usepackage{acl2014}
\usepackage{times}
\usepackage{url}
\usepackage{latexsym}

%\setlength\titlebox{5cm}

% You can expand the titlebox if you need extra space
% to show all the authors. Please do not make the titlebox
% smaller than 5cm (the original size); we will check this
% in the camera-ready version and ask you to change it back.


\title{Discourse}

\author{First Author \\
  Affiliation / Address line 1 \\
  Affiliation / Address line 2 \\
  Affiliation / Address line 3 \\
  {\tt email@domain} \\\And
  Second Author \\
  Affiliation / Address line 1 \\
  Affiliation / Address line 2 \\
  Affiliation / Address line 3 \\
  {\tt email@domain} \\}

\date{}

\begin{document}
\maketitle

\begin{abstract}
\end{abstract}

\section{Introduction}

Automatic coherence assessment has typically examined the effect of local discourse features on coherence. These features are typically sentence-to-sentence transitions of entities, discourse relations, or other sentential information. 
Additionally, this task is often framed as selecting the most coherent ordering of a set of sentences in a document, where the original ordering as written by a human is assumed to be the most coherent ordering.

\section{General Instructions}




% include your own bib file like this:
\bibliographystyle{acl}
\bibliography{acl2014}

\end{document}

%%% Local Variables: 
%%% mode: latex
%%% TeX-master: t
%%% End: 
