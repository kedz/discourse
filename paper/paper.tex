\documentclass[11pt]{article}
\usepackage{acl2014}

\usepackage{amssymb}
\usepackage{times}
\usepackage{url}
\usepackage{latexsym}
\usepackage{todonotes}

\newcommand{\Word}{w}
\newcommand{\Structs}{{\cal Y}}
\newcommand{\Sent}{s}
\newcommand{\IndexSet}{{\cal I}}
\newcommand{\Sents}{{\cal S}}
\newcommand{\Order}{\sigma}
\newcommand{\Enum}[1]{\{ 1 \ldots #1 \}}
\newcommand{\Set}[1]{\big\{ #1 \big\}}
\newcommand{\Bin}[1]{\{0,1\}^{#1}}
\newcommand{\Reals}{{\mathbb R}}

\newcommand{\Ver}{{\cal V}}
\newcommand{\Edges}{{\cal E}}


%\setlength\titlebox{5cm}

% You can expand the titlebox if you need extra space
% to show all the authors. Please do not make the titlebox
% smaller than 5cm (the original size); we will check this
% in the camera-ready version and ask you to change it back.

\usepackage{amsmath}
\DeclareMathOperator*{\argmax}{arg\,max}
\DeclareMathOperator*{\argmin}{arg\,min}


\title{Discourse}

\author{First Author \\
  Affiliation / Address line 1 \\
  Affiliation / Address line 2 \\
  Affiliation / Address line 3 \\
  {\tt email@domain} \\\And
  Second Author \\
  Affiliation / Address line 1 \\
  Affiliation / Address line 2 \\
  Affiliation / Address line 3 \\
  {\tt email@domain} \\}

\date{}

\begin{document}
\maketitle

\begin{abstract}
\end{abstract}

\section{Introduction}

Automatic coherence assessment has typically examined the effect of local discourse features on coherence. These features are typically sentence-to-sentence transitions of entities, discourse relations, or other sentential information. 
Additionally, this task is often framed as selecting the most coherent ordering of a set of sentences in a document, where the original ordering as written by a human is assumed to be the most coherent ordering.

% Pitchy

\section{Overview}

% High-level description of the problem.

% Ranking versus finding the best. 

% Graph, data analysis. 

\section{Background}

% Mathematical view of the problem.

For a given document, $x$ we assume  a bag of sentences $\Sents$ containing each sentence in a
document of size $|\Sents| = n$.  Each sentence $\Sent \in \Sents$
contains a list of words $w_1 \ldots w_m$ (for simplicity, we assume each sentence has the same length $m$). We will also assume that we
have part-of-speech tags, named entities, and the syntactic tree of
the sentence.


The aim of the ordering problem is to produce an ordering on the sentences in the documents, i.e. $\Order : \Sents \rightarrow \Enum{n}$ where $\Order$ is a bijective function. In order to predict an ordering we first define a more specific structure set.  

Define a transition index set as 

\[ \IndexSet = \Set{ (i, u, v) : u, v \in \Sents, i \in \Enum{n}, u \neq v } \]

And let the set of bigram orderings be defined as $\Structs \subset \Bin{\IndexSet}$.
The score of an ordering is given by a linear function  $f:\Bin{\IndexSet} \rightarrow \Reals$, and the decoding problem is to find 

\[\argmax_{y \in \Structs} f(y) = \argmax_{y \in \Structs} \theta^\top y \] 

\noindent where $\theta \in \Reals^{\IndexSet}$ is a parameter vector specific to the document $x$.

% In order to define the ordering constraints on $\Stucts$ considers a directed graph $(\Ver, \Edges)$ where $\Ver = \Enum{|\Sents|}$ and $\Edges = \Ver \times \Ver$. The weights $\theta$ can be interpreted as a edge weights on this directed graph. Define the set $\Structs$ to be all valid {\em tours} of this graph starting from any vertex.

% More specifically, we define $\Structs$ as 

% \[\Structs = \begin{cases} y \in \Bin{\IndexSet}: & \displaystyle \sum_{j = 1}^{|\Sents|} y_{ij} = 1 \ \  \forall i \in \Enum{\Sents}, \\ 
% & \displaystyle  \sum_{j = 1}^{|\Sents|} y_{ji} = 1 \ \ \forall i \in \Enum{\Sents} \\ 
% \end{cases} 
% \]    
 
% 
In practice, we redefine this problem using a lattice representation $(\Ver, \Edges)$ where $\Ver = \Enum{\Sents} \times \Enum{\Sents}$ and $\Edges$ defined as 

\[ {\Edges} = \Set{ ((i, j), (i + 1, k)): i,j,k \in \Enum{|\Sents|}, j \neq k  } \] 
 
And we define new constraints


\[\Structs' = \begin{cases} y : 
% & \displaystyle \sum_{i = 1}^n \sum_{v\in \Sents} y(i, u,v)  = 1 \ \  \forall u \in \Sents, \\ 
& \displaystyle \sum_{u, v\in  \Sents } y(1, u, v)  = 1,  \displaystyle\sum_{u, v \in \Sents } y(n, u, v)  = 1,\\

  & \displaystyle \sum_{u \in \Sents} y(i, u, v) = \sum_{w \in \Sents} y(i+ 1, v, w) \\ \ & \forall i \in \{1\ldots (n-1)\}, u, v \in \Sents, \\
  % & \displaystyle  \sum_{u \in \Sents} y(i, u, v) = 1 \ \ \forall i \in \Enum{n}, v \in \Sents \\
\end{cases} 
\]    

And we add the constraint 
\[\Structs = \begin{cases} y\in\Structs' :  & \displaystyle \sum_{i = 1}^n \sum_{u\in \Sents} y(i, u,v)  = 1 \ \  \forall v \in \Sents, \]



\todo[inline]{Show this} 
The decoding problem $\argmax_{y \in \Structs} f(y)$ is equivalent a directed traveling salesman problem over this graph \cite{}. 




% Each document




% decoding section.

\section{Features}

In this section we discuss how to compute the edge weights of the lattice. We will assume that $\theta(i, u, v) = w^\top \phi(x, (i, u, v))$ where $w \in \Reals^{d}$ is a parameter vector and $\phi: ({\cal X} \times \IndexSet) \rightarrow \Reals^{d}$ is a local feature function. 

\paragraph{Entity-Grid Features}

\newcite{bandl} propose a model of document  coherence using an Entity-Grid model.

% Define Entity-grid model

% Define how the features work for it. 

\paragraph{Sentence Topic Features}




\paragraph{Lexical Features}

\paragraph{Syntactic Features}


   


% Topic features

% Entity-grid discriminative


\section{Structured Training}


\section{Decoding}

\section{Related Work}

\section{Experiments}

\paragraph{Features}

% Exhaustive enumeration of feature
% description of you reimplementation of topics


\section{Results}

\paragraph{Data}

We use two corpora commonly used to assess document coherence \newcite{otherguys}.
The first corpus is a collection of ??? reports of earthquakes taken from the Associated Press Wire Service (APWS). The average document length is ???.  
The second is a set of airplane accident reports from the National Transportation Safety Board (NTSB)-- the average document length is ???.
These copora were taken from (\todo{link reginas page}). We use \cite{bandl}'s training and test splits for each corpus. We further divide the their training data into ???/??? split of training and development documents. All parameter tuning was done on the development set.

When comparing to \cite{bandl,louisandnenkova}, we use the set of document permutations also taken from (\todo{link again to Regina's page}).
% NTSB
% Stanford corenlp
% etc.

\paragraph{Implementation}


\paragraph{Baselines}

% S\&M
% B\&L
% beam
% ILP gurobi

\paragraph{Results}

% Describe each table.
% statistical significance

\begin{table*}
  \centering
  \begin{tabular}{|l|ll|ll|}
    \hline
    & \multicolumn{2}{c|}{Earthquake} & \multicolumn{2}{c|}{NTSB} \\
    & Tau & Bleu & Tau & Bleu \\
    \hline
    Model & & \\
    Model(Beam) & & \\
    S\&M & & \\
    B\&L All/no coref & & \\
    \hline
  \end{tabular}
\end{table*}


\begin{table*}
  \centering
  \begin{tabular}{|l|ll|ll|}
    \hline
    & \multicolumn{2}{c|}{Wiki Earthquake} \\
    & Tau & Bleu & Speed & \\
    \hline
    Model(Beam) & & \\
    \hline
  \end{tabular}
\end{table*}

\begin{table*}
  \centering
  \begin{tabular}{|l|lll|lll|}
    \hline
    & \multicolumn{3}{c|}{NTSB} &\multicolumn{3}{c|}{APWS} \\
    &Dev-$\tau$ & $\tau$ & Bleu &Dev-$\tau$ & $\tau$ & Bleu \\
    \hline
    beam-bigram fw-vbz-sx &0.43 &0.37 & & 0.53 & 0.40 & \\
    beam-bigram fw-vbz-sx-posq &0.41 &0.41 & & 0.65 & 0.45 & \\
    beam-bigram fw-vbz-sx-tpc &\textbf{0.60} &\textbf{0.59} & & 0.73 & 0.60 & \\  
    beam-bigram fw-vbz-sx-posq-trw20 &0.59 &0.59 & & 0.75 & 0.65 & \\
    beam-bigram fw-vbz-sx-posq-tpc  &0.56 &0.56 & & \textbf{0.75}  & \textbf{0.66} & \\
    beam-bigram fw-vbz-sx-tpc-trw20 &0.59 &0.59 & & 0.74 & 0.64 & \\
    beam-bigram fw-vbz-sx-posq-tpc-trw20 &0.60 &0.56 & &  0.73  & 0.63 & \\
    \hline 
    beam-bigram vbz-posq-tpc-trw20 &0.55 &0.55 & & 0.67 & 0.58 & \\ 
    beam-bigram sx-posq-tpc-trw20 &0.56 &0.52 & & 0.73 & 0.60 & \\
    beam-bigram fw-posq-tpc-trw20 &\textbf{0.59} &\textbf{0.53} & & \textbf{0.77} & \textbf{0.66} & \\
    \hline
    beam-bigram vbz-sx-posq-tpc-trw20 &0.57 &0.55 & & 0.68 &0.62 & \\
    beam-bigram fw-vbz-posq-tpc-trw20 &0.55 &0.59 & & \textbf{0.75} &\textbf{0.66}& \\
    beam-bigram fw-sx-posq-tpc-trw20  &\textbf{0.57} &\textbf{0.58} & & 0.74 &0.66 & \\
    \hline 
    \hline
    loss tau \\
    loss bigram \\
    loss unigram \\
    loss 0/1 \\
    \hline
  \end{tabular}
\end{table*}

\pargraph{Analysis}
Error analysis.

\section{Conclusion}

We did this.

Next we do this.

% \begin{table}
%   \centering
%   \begin{tabular}{l|ll}
%      & Tau & Bigram \\
%     Model & & \\
%     Model(Beam) & & \\
%     R\&B All/no coref & & \\
%   \end{tabular}
% \end{table}

%%% Local Variables:
%%% mode: latex
%%% TeX-master: "paper.tex"
%%% End:


% include your own bib file like this:
\bibliographystyle{acl}
\bibliography{acl2014}

\end{document}

%%% Local Variables: 
%%% mode: latex
%%% TeX-master: t
%%% End: 
