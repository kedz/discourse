\documentclass[11pt]{article}
\usepackage{acl2014}
\usepackage{times}
\usepackage{url}
\usepackage{latexsym}

%\setlength\titlebox{5cm}

% You can expand the titlebox if you need extra space
% to show all the authors. Please do not make the titlebox
% smaller than 5cm (the original size); we will check this
% in the camera-ready version and ask you to change it back.


\title{Discourse}

\author{First Author \\
  Affiliation / Address line 1 \\
  Affiliation / Address line 2 \\
  Affiliation / Address line 3 \\
  {\tt email@domain} \\\And
  Second Author \\
  Affiliation / Address line 1 \\
  Affiliation / Address line 2 \\
  Affiliation / Address line 3 \\
  {\tt email@domain} \\}

\date{}

\begin{document}
\maketitle

\begin{abstract}
\end{abstract}

\section{Introduction}

Automatic coherence assessment has typically examined the effect of local discourse features on coherence. These features are typically sentence-to-sentence transitions of entities, discourse relations, or other sentential information.
Additionally, this task is often framed as selecting the most coherent ordering of a set of sentences in a document, where the original ordering as written by a human is assumed to be the most coherent ordering.

\section{General Instructions}



\section{Experiments}

\paragraph{Features}

% Exhaustive enumeration of feature
% description of you reimplementation of topics


\section{Results}

\paragraph{Data}

We use two corpora commonly used to assess document coherence \newcite{otherguys}.
The first corpus is a collection of ??? reports of earthquakes taken from the Associated Press Wire Service (APWS). The average document length is ???.  
The second is a set of airplane accident reports from the National Transportation Safety Board (NTSB)-- the average document length is ???.
These copora were taken from (\todo{link reginas page}). We use \cite{bandl}'s training and test splits for each corpus. We further divide the their training data into ???/??? split of training and development documents. All parameter tuning was done on the development set.

When comparing to \cite{bandl,louisandnenkova}, we use the set of document permutations also taken from (\todo{link again to Regina's page}).
% NTSB
% Stanford corenlp
% etc.

\paragraph{Implementation}


\paragraph{Baselines}

% S\&M
% B\&L
% beam
% ILP gurobi

\paragraph{Results}

% Describe each table.
% statistical significance

\begin{table*}
  \centering
  \begin{tabular}{|l|ll|ll|}
    \hline
    & \multicolumn{2}{c|}{Earthquake} & \multicolumn{2}{c|}{NTSB} \\
    & Tau & Bleu & Tau & Bleu \\
    \hline
    Model & & \\
    Model(Beam) & & \\
    S\&M & & \\
    B\&L All/no coref & & \\
    \hline
  \end{tabular}
\end{table*}


\begin{table*}
  \centering
  \begin{tabular}{|l|ll|ll|}
    \hline
    & \multicolumn{2}{c|}{Wiki Earthquake} \\
    & Tau & Bleu & Speed & \\
    \hline
    Model(Beam) & & \\
    \hline
  \end{tabular}
\end{table*}

\begin{table*}
  \centering
  \begin{tabular}{|l|lll|lll|}
    \hline
    & \multicolumn{3}{c|}{NTSB} &\multicolumn{3}{c|}{APWS} \\
    &Dev-$\tau$ & $\tau$ & Bleu &Dev-$\tau$ & $\tau$ & Bleu \\
    \hline
    beam-bigram fw-vbz-sx &0.43 &0.37 & & 0.53 & 0.40 & \\
    beam-bigram fw-vbz-sx-posq &0.41 &0.41 & & 0.65 & 0.45 & \\
    beam-bigram fw-vbz-sx-tpc &\textbf{0.60} &\textbf{0.59} & & 0.73 & 0.60 & \\  
    beam-bigram fw-vbz-sx-posq-trw20 &0.59 &0.59 & & 0.75 & 0.65 & \\
    beam-bigram fw-vbz-sx-posq-tpc  &0.56 &0.56 & & \textbf{0.75}  & \textbf{0.66} & \\
    beam-bigram fw-vbz-sx-tpc-trw20 &0.59 &0.59 & & 0.74 & 0.64 & \\
    beam-bigram fw-vbz-sx-posq-tpc-trw20 &0.60 &0.56 & &  0.73  & 0.63 & \\
    \hline 
    beam-bigram vbz-posq-tpc-trw20 &0.55 &0.55 & & 0.67 & 0.58 & \\ 
    beam-bigram sx-posq-tpc-trw20 &0.56 &0.52 & & 0.73 & 0.60 & \\
    beam-bigram fw-posq-tpc-trw20 &\textbf{0.59} &\textbf{0.53} & & \textbf{0.77} & \textbf{0.66} & \\
    \hline
    beam-bigram vbz-sx-posq-tpc-trw20 &0.57 &0.55 & & 0.68 &0.62 & \\
    beam-bigram fw-vbz-posq-tpc-trw20 &0.55 &0.59 & & \textbf{0.75} &\textbf{0.66}& \\
    beam-bigram fw-sx-posq-tpc-trw20  &\textbf{0.57} &\textbf{0.58} & & 0.74 &0.66 & \\
    \hline 
    \hline
    loss tau \\
    loss bigram \\
    loss unigram \\
    loss 0/1 \\
    \hline
  \end{tabular}
\end{table*}

\pargraph{Analysis}
Error analysis.

\section{Conclusion}

We did this.

Next we do this.

% \begin{table}
%   \centering
%   \begin{tabular}{l|ll}
%      & Tau & Bigram \\
%     Model & & \\
%     Model(Beam) & & \\
%     R\&B All/no coref & & \\
%   \end{tabular}
% \end{table}

%%% Local Variables:
%%% mode: latex
%%% TeX-master: "paper.tex"
%%% End:

% include your own bib file like this:
\bibliographystyle{acl}
\bibliography{acl2014}

\end{document}

%%% Local Variables:
%%% mode: latex
%%% TeX-master: t
%%% End:
